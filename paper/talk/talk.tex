\documentclass[slideopt,A4,showboxes,svgnames]{beamer}

%% list of packages here
\usepackage[absolute,showboxes,overlay]{textpos}
\usepackage{booktabs}
\usepackage{mathtools}
\usepackage{amsmath}
\usepackage{amssymb}
\usepackage{pifont}
\usepackage{multicol}
\usepackage{multirow}
\usepackage{array, makecell}
\usepackage{blkarray}
\usepackage{nccmath}
\usepackage[linesnumbered,ruled,vlined]{algorithm2e}
\usepackage[backend=biber, style=authoryear-icomp,maxbibnames=9,maxcitenames=2,uniquelist=false]{biblatex}
\renewcommand*{\nameyeardelim}{\addcomma\addspace}
\addbibresource{../references.bib}

\usepackage{theme/beamerthemeinria}
%\input{../mathdef}
\input{./mathdef}
\input{theme/style}

\title[Robust-Adaptive Control]{Robust-Adaptive\\Interval Predictive Control\\ for Linear Uncertain Systems}
\subtitle{Sous-titre}
\date[date]{date}
\author[Edouard Leurent]{\textbf{Edouard Leurent}\inst{1,2},\\
	Odalric-Ambrym Maillard\inst{1},\\
	Denis Efimov\inst{1}}
\institute{
	\inst{1} Univ. Lille, Inria, CNRS, \\ ~Centrale Lille, UMR 9189 – CRIStAL,\\
	\inst{2} Renault Group}

\begin{document}

\begin{frame}
    \titlepage
\end{frame}

\begin{frame}{Motivation}

\begin{itemize}[<+->]
	\item Real-world systems face a severe {\red model uncertainty}\\~
	\item Ensuring {\green safety} requires maintaining strict \\{\red state and control constraints}
\end{itemize}
\end{frame}

\begin{frame}{Setting}
\begin{exampleblock}{Objective}
\begin{itemize}
	\item Stabilization of a linear system \useshortskip
	\begin{align*}
	{\dot{x}(t)} &= \textcolor<4>{orange}{A(\theta)}{x(t)} + B u(t) + \textcolor<2-4>{normalBlockTitleTextColor}{\omega(t)},
	\end{align*}
	\item Constraints on state $x\in\X$, control $u\in\U$.
\end{itemize}
\end{exampleblock}
\pause
\begin{alertblock}{Subject to uncertainty}
	\begin{itemize}[<+->]
		\item Bounded disturbance ${\red \underline\omega(t) \leq \omega(t) \leq \overline\omega(t)}$;
		\item Bounded measurement noise ${\red \underline\eta(t) \leq \eta(t) \leq \overline\eta(t)}$,
		\useshortskip
		\begin{equation*} y(t) = \begin{bmatrix}x(t) & \dot x(t)\end{bmatrix}^\top + {\red \eta(t)};\end{equation*}
		\item Parametric uncertainty in the state matrix \useshortskip
	\begin{align*}
	{\orange A(\theta)} = A + \sum_{i=1}^d {\orange \theta_i}\phi_i\text{ is unknown.}
	\end{align*}
	\end{itemize}\vspace*{-0.5cm}
\end{alertblock}
\end{frame}

\begin{frame}{Related work}

\begin{alertblock}{Adaptive Model Predictive Control (MPC)}
	See \textit{e.g.} [\cite{Fukushima2007,Adetola2009,Adetola2011,Aswani2013,Vicente2019}]
	\pause
	\begin{enumerate}[<+->]
		\item Estimation algorithm to evaluate the model uncertainty
		\item MPC to enforce constraints during transients
	\end{enumerate}
\end{alertblock}
\pause[\thebeamerpauses]
\begin{exampleblock}{Our work}
	\begin{itemize}[<+->]
		\item A simple solution using \alert{intervals}, based on a \alert{recent predictor} [\cite{leurent2019interval}]
		\item Composed of 4 building blocks \begin{enumerate}
			\item Model Estimation
			\item State Prediction
			\item Robust Stabilization
			\item Robust Constraint Satisfaction
		\end{enumerate}
	\end{itemize}
\end{exampleblock}
\end{frame}

\begin{frame}{Step 1: Model Estimation}

\textbf{Objective}: Derive a tighter admissible set ${\orange \hat{\Theta}} \subsetneq \Theta$.

\pause
\begin{assumption}[Persistence of Excitation (PE)]
	\label{assu:PE}
	The features $\Phi(t)=[\phi_{1}y_{1}(t)\dots\phi_{d}y_{1}(t)]\in\Real^{p\times d}$ verify
	\[\forall t\geq 0,\;
	\int_{t}^{t+\ell}\Phi^{\top}(s)\Phi(s)ds\ge\vartheta I_{d}.
	\]
\end{assumption}
\pause
Define the OLS estimate\begin{equation*}
\hat{\theta}(t)=
\left(\int_{0}^{t}\Phi^{\top}(s)\Phi(s)ds\right)^{-1}\int_{0}^{t}\Phi^{\top}(s)y(s)ds,\, t\geq\ell,
\end{equation*}
\end{frame}

\begin{frame}{Step 1: Model Estimation}
\begin{proposition}[Admissible values $\hat{\Theta}(t)\subseteq\Theta$]
	\begin{equation*}
	\theta \in {\orange \hat{\Theta}(t)}=\Theta\bigcap_{\tau\in[\ell,t]}\{\tilde{\theta}\in\Real^{d}:|\hat{\theta}(\tau)-\tilde{\theta}|\leq\Delta\theta(X)\}.\label{eq:set_LS}
	\end{equation*}
\end{proposition}
${\orange \hat{\Theta}(t)}$ \alert{shrinks} as we get more observations $y(\tau),\,\tau\leq t$.
\begin{center}
	\includegraphics[trim={0 1cm 0 1cm}, clip,width=0.43\linewidth]{img/ellipsoid}
\end{center}
\end{frame}

\begin{frame}{Step 2: State Prediction}
Given our sources of uncertainty ${\red [\underline\omega,\overline\omega]}$, ${\red [\underline\eta,\overline\eta]}$, $\orange \hat{\Theta}(t)$, \\
we want to bound the set of reachable states: 
\begin{center}
\includegraphics[width=0.35\linewidth]{img/interval-hull}
\end{center}
within an interval \useshortskip
\begin{equation*}
{\blue \ux(s)}\leq x(s)\leq{\blue \ox(s)},\quad\forall s\geq t.
\end{equation*}\pause
\begin{itemize}
	\vspace*{-0.5cm}
	\item[\incarrow] To derive this predictor, two possible \alert{representations of the uncertainty} over $A(\theta)$ can be used.
\end{itemize}
\end{frame}


\begin{frame}{Step 2: Interval Uncertainty}
For all $\theta\in\hat{\Theta}(t)$,
{\orange$\quad\underline{A}\leq A(\theta)\leq\overline{A}
$}.
\begin{proposition}[Simple predictor of \cite{Efimov2012}]
{\small
\begin{align*}
{\blue \dot{\underline{x}}(s)} & = {\orange \underline{A}^{+}}\underline{x}^{+}(s)-{\orange\overline{A}^{+}}\underline{x}^{-}(s)-{\orange\underline{A}^{-}}\overline{x}^{+}(s)+{\orange\overline{A}^{-}}\overline{x}^{-}(s) +Bu(s)+\underline{\omega}(s),\\
{\blue \dot{\overline{x}}(s)} & = {\orange \overline{A}^{+}}\overline{x}^{+}(s)-{\orange\underline{A}^{+}}\overline{x}^{-}(s)-{\orange\overline{A}^{-}}\underline{x}^{+}(s)+{\orange\underline{A}^{-}}\underline{x}^{-}(s)
+Bu(s)+\overline{\omega}(s), \\
&  \underline{x}(t)=y_{1}(t)-\underline{\nu}_{1}(t),\;\overline{x}(t)=y_{1}(t)+\overline{\nu}_{1}(t), 
\end{align*}
}
ensures the inclusion property.
\end{proposition}
\begin{center}
	\includegraphics[width=0.6\linewidth]{img/observer}
	\vspace*{-1cm}
\end{center}
\end{frame}

\begin{frame}{Step 2: Polytopic Uncertainty} $\forall\theta\in\hat{\Theta}(t)$, ${\orange A(\theta)}\in\left\{ {\orange A_{0}+\sum_{i=1}^{2^{d}}\alpha_{i}\Delta A_{i}}\right\}$ with ${\orange A_0}$ {\red Metzler}.
\begin{proposition}[Enhanced predictor of \cite{leurent2019interval}]
	{\small
	\begin{align*}
{\blue \dot{\underline{x}}(s)} & = {\orange A_{0}}\underline{x}(s)-{\orange \Delta A_{+}}\underline{x}^{-}(s)-{\orange \Delta A_{-}}\overline{x}^{+}(s)
+Bu(s)+\underline{\omega}(s) \\
{\blue \dot{\overline{x}}(s)} & = {\orange A_{0}}\overline{x}(s)+{\orange \Delta A_{+}}\overline{x}^{+}(s)+{\orange \Delta A_{-}}\underline{x}^{-}(s)\label{eq:interval-predictor}
+Bu(s)+\overline{\omega}(s) \\
& \underline{x}(t)=y_{1}(t)-\underline{\nu}_{1}(t),\;\overline{x}(t)=y_{1}(t)+\overline{\nu}_{1}(t),\nonumber 
	\end{align*}}
	ensures the inclusion property.
\end{proposition}
\begin{center}
\includegraphics[width=0.6\linewidth]{img/predictor}
\vspace*{-1cm}
\end{center}
\end{frame}

\begin{frame}{Step 3: Robust Stabilization}
%Now that we have 
%$
%{\blue \ux(s)}\leq x(s)\leq{\blue \ox(s)},\quad\forall s\geq t,
%$

We want to \alert{stabilize} $$ {\blue \xi(t) = \begin{bmatrix}\underline{x}(t) \\ \overline{x}(t)\end{bmatrix}}$$


%
%Its dynamics can be written as \begin{equation}
%\dot{\xi}(t)=\cA_{0}\xi (t)+\cA_{1}\xi^{+}(t)+\cA_{2}\xi^{-}(t)+\cB u(t)+\delta(t),
%\end{equation}
%with $\delta(t)=\begin{bmatrix}
%\underline{\omega}(t) & \overline{\omega}(t)
%\end{bmatrix}^\top\in\Real^{2p}
%$
\begin{exampleblock}{Controller design}
We propose to look for a {\green control} in the form
\begin{equation*}
u(t)={\green K_{0}}{\blue \xi(t)}+{\green K_{1}}{\blue \xi^{+}(t)}+{\green K_{2}}{\blue \xi^{-}(t)}+ {\green S}{\red \begin{bmatrix}\underline{\omega}(t) & \overline{\omega}(t)\end{bmatrix}}^\top
\end{equation*}
\end{exampleblock}
\begin{flushright}
How to choose the gains?
\end{flushright}
\end{frame}

\begin{frame}{Step 3: Robust Stabilization}

The gains ${\green K_{0},K_{1},K_{2}}$ have to respect the following restriction:
\begin{theorem}[ISS Stability]
	
	If there exist diagonal matrices $P$, $Q$, $Q_{+}$,
	$Q_{-}$, $Z_{+}$, $Z_{-}$, $\Psi_{+}$, $\Psi_{-}$, $\Psi$, $\Gamma\in\Real^{2p\times2p}$
	such that:
	\scriptsize
	\begin{gather*}
	{\green \Upsilon\preceq0},\;P+\min\{Z_{+},Z_{-}\}>0,\;\Gamma>0,\\
	Q+\min\{Q_{+},Q_{-}\}+2\min\{\Psi_{+},\Psi_{-}\}>0,
	\end{gather*}
	\vspace*{-0.8cm}\tiny \begin{gather*}
	\text{where }\quad {\green \Upsilon}=\left[\begin{array}{cccc}
	\Upsilon_{11} & \Upsilon_{12} & \Upsilon_{13} & P\\
	\Upsilon_{12}^{\top} & \Upsilon_{22} & \Upsilon_{23} & Z_{+}\\
	\Upsilon_{13}^{\top} & \Upsilon_{23}^{\top} & \Upsilon_{33} & -Z_{-}\\
	P & Z_{+} & -Z_{-} & -\Gamma
	\end{array}\right], \\
	\Upsilon_{11}=\cD_{0}^{\top}P+P\cD_{0}+Q,\;\Upsilon_{12}=\cD_{0}^{\top}Z_{+}+P\cD_{1}+\Psi_{+},\\
	\Upsilon_{13}=P\cD_{2}-\cD_{0}^{\top}Z_{-}-\Psi_{-},\;\Upsilon_{22}=Z_{+}\cD_{1}+\cD_{1}^{\top}Z_{+}+Q_{+},\\
	\Upsilon_{23}=Z_{+}\cD_{2}-\cD_{1}^{\top}Z_{-}+\Psi,\;\Upsilon_{33}=-Z_{-}\cD_{2}-\cD_{2}^{\top}Z_{-}+Q_{-},\\
	{\cD_{i}}=\cA_{i}+\cB {\green K_{i}},\;i\in[3]
	\end{gather*}
	\normalsize
	then ${\blue \xi(t)}$ is \alert{input-to-state stable} with respect
	to ${\red \underline{\omega},\overline{\omega}}$.
\end{theorem}
\end{frame}

\begin{frame}{Are we done?}
\begin{exampleblock}{Basin of attraction}
All trajectories are asymptotically attracted into \useshortskip
\begin{equation*}
\X_{f}=\left\{\xi\in\Real^{2p}:V({\blue\xi})\leq\alpha^{-1}\sup_{t\geq0}|{\red \tilde{\delta}^{\top}(t)}\Gamma{\red \tilde{\delta}(t)}|\right\}
\end{equation*}
\end{exampleblock}

\pause
\begin{alertblock}{But...}
	Constraints $x\in\X,\,u\in\U$ are not enforced.
\end{alertblock}
\pause
\begin{exampleblock}{Idea}
	\begin{enumerate}
		\item Assume the constraints hold asymptotically: $$\X_{f}\subset\X^{2}\text{  and  }u(\X_f) \subset\U$$
		\item Find an open-loop control to reach $\X_f$ under the constraints $x\in\X,\,u\in\U$.
	\end{enumerate}
\end{exampleblock}
\end{frame}

\begin{frame}{Step 4: Robust Constraint Satisfaction}

\begin{enumerate}[<+->]
	\item Find the controls
	\begin{gather*}
	\cU=\argmin_{u:[t_{i},t_{i}+T]\to\Real^{q}} 	\int_{t_{i}}^{t_{i}+T}\xi^{\top}(s)W_{1}\xi(s)+u^{\top}(s)W_{2}u(s)ds
	\end{gather*}
	subject to: 
	\begin{itemize}
		\item $\xi(s)\in\X^{2}$ and $u(s)\in\U$;
		\item $\xi(t_{i}+T)\in\X_{f}$.
	\end{itemize}
	\item For $t\in[t_{i},t_{i}+\tau)$ select
	\begin{equation*}
	u(t)=\begin{cases}
	\cU(t) & \xi(t_{i})\notin\X_{f}\\
	{\green K_{0}}{\blue \xi(t)}+{\green K_{1}}{\blue \xi^{+}(t)}+{\green K_{2}}{\blue \xi^{-}(t)}+ {\green S}{\red \delta(t)} & \xi(t_{i})\in\X_{f}
	\end{cases},\label{eq:control}
	\end{equation*}
	where $K_{0},K_{1},K_{2}$ are solutions of the stability LMI.
\end{enumerate}
\end{frame}

\begin{frame}{Step 4: Robust Constraint Satisfaction}
\begin{theorem}[Main result]
	\label{th:MPC} Under the previous assumptions, the closed-loop
	system given by \begin{itemize}
		\item the {system dynamics},
		\item the {interval predictor},
		\item the {interval feedback},
		\item the {robust MPC}
	\end{itemize}
	has the following properties:
	\pause
	\begin{enumerate}[<+->]
		\item \alert{Input-to-state stability} for ${\blue \underline{x},\;\overline{x}}$ and practical
		input-to-state stability for $x$ with respect to ${\red \underline{\omega},\overline{\omega}}$
		in the terminal set $\X_{f}$; 
		\item \alert{Recursive feasibility} with reaching $\X_{f}$ in a finite time; 
		\item \alert{Constraint satisfaction}.
	\end{enumerate}
\end{theorem}
\end{frame}

\begin{frame}{Numerical Experiment}
Autonomous vehicle with {\orange unknown tire friction}
\pause
\begin{itemize}[<+->]
	\item \textbf{State}: \useshortskip $$x = \begin{bmatrix} {p_y} & {\psi} & {v_y} & {r} \end{bmatrix}^\top $$
	\item \textbf{Dynamics}: Dynamical Bicycle Model [\cite{awan2014compensation}] with unknown tire friction coefficients ${\orange \theta = \begin{bmatrix} C_{\alpha_f} & C_{\alpha_r}\end{bmatrix}^\transp}$
	\useshortskip \[
	\small
	A = \begin{bmatrix}
	0 & v_x & 1 & 0 \\
	0 & 0 & 0 & 1 \\
	0 & 0 & 0 & - v_x \\
	0 & 0 & 0 & 0
	\end{bmatrix},\quad
	B =
	\begin{bmatrix}
	0 \\
	0 \\
	\frac{2}{m} \\
	\frac{a}{I_z}
	\end{bmatrix},
	\]
	\[
	\phi = \frac{-2}{m v_x I_z}\left[\begin{bmatrix}
	0 & 0 & 0 & 0 \\
	0 & 0 & 0 & 0 \\
	0 & 0 & I_z & a I_z \\
	0 & 0 & a m & a^2 m \\
	\end{bmatrix},\begin{bmatrix}
	0 & 0 & 0 & 0 \\
	0 & 0 & 0 & 0 \\
	0 & 0 & I_z & -b I_z \\
	0 & 0 & - bm & b^2 m \\
	\end{bmatrix}\right].
	\]
\end{itemize}

\end{frame}

\begin{frame}
In this video, we show the predicted state intervals used to derive an open-loop transient control sequence, which robustly brings the system in a vicinity of the lane centreline. Once this region is reached, we switch to closed-loop control to robustly stabilize the system.
\end{frame}

\begin{frame}
\centering \LARGE Thank You!\\[1cm]
\large \emph{I am looking for a postdoctoral position.}
\end{frame}


\end{document}
